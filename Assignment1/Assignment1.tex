\documentclass[journal,12pt,twocolumn]{IEEEtran}
%
\usepackage{setspace}
\usepackage{gensymb}
%\doublespacing
\singlespacing

\usepackage{graphicx}
\usepackage[cmex10]{amsmath}
\usepackage{amsmath,amsthm}
\usepackage{mathrsfs}
\usepackage{txfonts}
\usepackage{stfloats}
\usepackage{bm}
\usepackage{cite}
\usepackage{cases}
\usepackage{subfig}

\usepackage{longtable}
\usepackage{multirow}
\usepackage{commath}
\usepackage{enumitem}
\usepackage{mathtools}
\usepackage{steinmetz}
\usepackage{tikz}
\usepackage{circuitikz}
\usepackage{verbatim}
\usepackage{tfrupee}
\usepackage[breaklinks=true]{hyperref}

\usepackage{tkz-euclide}

\usetikzlibrary{calc,math}
\usepackage{listings}
    \usepackage{color}                                            
    \usepackage{array}                                            
    \usepackage{longtable}                                        
    \usepackage{calc}                                             
    \usepackage{multirow}                                         
    \usepackage{hhline}                                           
    \usepackage{ifthen}                                           
    \usepackage{lscape}     
\usepackage{multicol}
\usepackage{chngcntr}

\DeclareMathOperator*{\Res}{Res}

\renewcommand\thesection{\arabic{section}}
\renewcommand\thesubsection{\thesection.\arabic{subsection}}
\renewcommand\thesubsubsection{\thesubsection.\arabic{subsubsection}}

\renewcommand\thesectiondis{\arabic{section}}
\renewcommand\thesubsectiondis{\thesectiondis.\arabic{subsection}}
\renewcommand\thesubsubsectiondis{\thesubsectiondis.\arabic{subsubsection}}

\hyphenation{op-tical net-works semi-conduc-tor}
\def\inputGnumericTable{}                                 

\lstset{
%language=C,
frame=single, 
breaklines=true,
columns=fullflexible
}
\lstset{
%language=TeX,
frame=single, 
breaklines=true
}

\begin{document}

\newtheorem{theorem}{Theorem}[section]
\newtheorem{problem}{Problem}
\newtheorem{proposition}{Proposition}[section]
\newtheorem{lemma}{Lemma}[section]
\newtheorem{corollary}[theorem]{Corollary}
\newtheorem{example}{Example}[section]
\newtheorem{definition}[problem]{Definition}

\newcommand{\BEQA}{\begin{eqnarray}}
\newcommand{\EEQA}{\end{eqnarray}}
\newcommand{\define}{\stackrel{\triangle}{=}}
\bibliographystyle{IEEEtran}
\providecommand{\mbf}{\mathbf}
\providecommand{\pr}[1]{\ensuremath{\Pr\left(#1\right)}}
\providecommand{\qfunc}[1]{\ensuremath{Q\left(#1\right)}}
\providecommand{\sbrak}[1]{\ensuremath{{}\left[#1\right]}}
\providecommand{\lsbrak}[1]{\ensuremath{{}\left[#1\right.}}
\providecommand{\rsbrak}[1]{\ensuremath{{}\left.#1\right]}}
\providecommand{\brak}[1]{\ensuremath{\left(#1\right)}}
\providecommand{\lbrak}[1]{\ensuremath{\left(#1\right.}}
\providecommand{\rbrak}[1]{\ensuremath{\left.#1\right)}}
\providecommand{\cbrak}[1]{\ensuremath{\left\{#1\right\}}}
\providecommand{\lcbrak}[1]{\ensuremath{\left\{#1\right.}}
\providecommand{\rcbrak}[1]{\ensuremath{\left.#1\right\}}}
\theoremstyle{remark}
\newtheorem{rem}{Remark}
\newcommand{\sgn}{\mathop{\mathrm{sgn}}}
\providecommand{\abs}[1]{\(\left\vert#1\right\vert\)}
\providecommand{\res}[1]{\Res\displaylimits_{#1}} 
\providecommand{\norm}[1]{\(\left\lVert#1\right\rVert\)}
%\providecommand{\norm}[1]{\lVert#1\rVert}
\providecommand{\mtx}[1]{\mathbf{#1}}
\providecommand{\mean}[1]{E\(\left[ #1 \right]\)}
\providecommand{\fourier}{\overset{\mathcal{F}}{ \rightleftharpoons}}
%\providecommand{\hilbert}{\overset{\mathcal{H}}{ \rightleftharpoons}}
\providecommand{\system}{\overset{\mathcal{H}}{ \longleftrightarrow}}
	%\newcommand{\solution}[2]{\textbf{Solution:}{#1}}
\newcommand{\solution}{\noindent \textbf{Solution: }}
\newcommand{\cosec}{\,\text{cosec}\,}
\providecommand{\dec}[2]{\ensuremath{\overset{#1}{\underset{#2}{\gtrless}}}}
\newcommand{\myvec}[1]{\ensuremath{\begin{pmatrix}#1\end{pmatrix}}}
\newcommand{\mydet}[1]{\ensuremath{\begin{vmatrix}#1\end{vmatrix}}}
%\numberwithin{equation}{section}
\numberwithin{equation}{subsection}
%\numberwithin{problem}{section}
%\numberwithin{definition}{section}
\makeatletter
\@addtoreset{figure}{problem}
\makeatother
\let\StandardTheFigure\thefigure
\let\vec\mathbf
%\renewcommand{\thefigure}{\theproblem.\arabic{figure}}
\renewcommand{\thefigure}{\theproblem}
%\setlist[enumerate,1]{before=\renewcommand\theequation{\theenumi.\arabic{equation}}
%\counterwithin{equation}{enumi}
%\renewcommand{\theequation}{\arabic{subsection}.\arabic{equation}}
\def\putbox#1#2#3{\makebox[0in][l]{\makebox[#1][l]{}\raisebox{\baselineskip}[0in][0in]{\raisebox{#2}[0in][0in]{#3}}}}
     \def\rightbox#1{\makebox[0in][r]{#1}}
     \def\centbox#1{\makebox[0in]{#1}}
     \def\topbox#1{\raisebox{-\baselineskip}[0in][0in]{#1}}
     \def\midbox#1{\raisebox{-0.5\baselineskip}[0in][0in]{#1}}
\vspace{3cm}
\title{Assignment 1}
\author{Addagalla Satyanarayana}
\maketitle
\newpage
%\tableofcontents
\bigskip
\renewcommand{\thefigure}{\theenumi}
\renewcommand{\thetable}{\theenumi}
\begin{abstract}
This document explains how to find a line perpendicular to 2 lines and passing through a point.
\end{abstract}
Download the python code from 
%
\begin{lstlisting}
https://github.com/AddagallaSatyanarayana/AI5006/tree/master/Assignment1
\end{lstlisting}
%
and latex-tikz codes from 
%
\begin{lstlisting}
https://github.com/AddagallaSatyanarayana/AI5006/tree/master/Assignment1/Assignment1.tex
\end{lstlisting}
%
\section{Problem}
Find the vector equation of the line passing through the point
$\begin{psmallmatrix}1 \\2 \\-4	\end{psmallmatrix}$
and perpendicular to the two lines
   	$\frac{x-8}{3} = \frac{y+19}{-16}= \frac{z-10}{7}$  and
	$\frac{x-15}{3} = \frac{y-29}{8}= \frac{z-5}{-5}$
	
	
\section{Explanation}
Equation of a line $\vec{l}$ passing through the point $\vec{a}$ and parallel to the line $\vec{n}$ is given by:
\begin{align}
 \vec{l} =\vec{a} + \lambda\vec{n} 
\end{align}
where $\lambda$ is some constant.
Since the line passes through 
	$\begin{psmallmatrix}1 \\2\\-4\end{psmallmatrix} $,
$\vec{a}$ = (1  2 -4)

\begin{align}   
\vec{l_1} =\begin{pmatrix} 8\\ -19\\ 10\end{pmatrix} + \lambda_1
	\begin{pmatrix} 3 \\-16\\ 7\end{pmatrix} ;
\end{align}

\begin{align} 
 \vec{l_2} =\begin{pmatrix}15\\ 29\\ 5\end{pmatrix} + \lambda_2
	\begin{pmatrix} 3 \\8\\ -5\end{pmatrix} 
\end{align}
\section{Solution}
	Let $\vec{n}$ be the normal vector to both lines. If $\vec{m_1}$ and $\vec{m_2}$ are the direction vectors of the lines,then
\par

\begin{align}
 \vec{m_1}^T  \vec{n} = 0   
\end{align}
\begin{align}
\vec{m_2}^T\vec{n} = 0    
\end{align}
Let $A$ be a point on line $l_1$ and $B$ be point on the line $l_2$.\par
The vector passing through the points $A$ and $B$ will be
\begin{align}
	\vec{A-B}=\vec{x_1-x_2}+\myvec{\vec{m_1} & -\vec{m_2}}\myvec{\lambda_1\\\lambda_2}
\end{align}
The vectors $\vec{m_1}$,$\vec{m_2}$ are both perpendicular to the line ${AB}$.So the dot product of $\vec{m_1}$,$\vec{m_2}$ with the line ${AB}$ is zero.\par
The dot product of $\vec{m_1}$ with the line ${AB}$ is
\begin{align}
	\vec{m_1}^T\brak{\vec{A-B}}=0\label{eq:32}\\
	\vec{m_1^T}(\vec{x_1-x_2})+\vec{m_1^T}(\vec{m_1} & -\vec{m_2})\begin{pmatrix}\lambda_1 \\ \lambda_2\end{pmatrix}=0
\end{align}
The dot product of $\vec{m_2}$ with the line ${AB}$ is
\begin{align}
	\vec{m_2}^T\brak{\vec{A-B}}=0\label{eq:34}\\
	\vec{m_2^T}(\vec{x_1-x_2})+\vec{m_2^T}(\vec{m_1} & -\vec{m_2})\myvec{\lambda_1 \\ \lambda_2}=0
\end{align}
Let the matrix $\vec{M}$ be
\begin{align}
	\vec{M}=\myvec{\vec{m_1}^T\\\vec{m_2}^T}
\end{align}
Combining the equations in matrix form,using equation we get
\begin{align}
	\vec{M}\vec{M}^T\myvec{\lambda_1\\-\lambda_2}+\vec{M}\vec{(x_1-x_2)}=0
\end{align}

\begin{align}
	\vec{M}\vec{M}^T\begin{pmatrix}\lambda_1\\-\lambda_2\end{pmatrix}=\vec{M}\vec{(x_2-x_1)}
\end{align}
\begin{align}
	\myvec{\vec{m_1}^T\vec{m_1} & \vec{m_1}^T\vec{m_2}\\\vec{m_2}^T\vec{m_1} & \vec{m_2}^T\vec{m_2}}\myvec{\lambda_1\\-\lambda_2}=\myvec{\vec{m_1}^T\vec{(x_2-x_1)}\\\vec{m_2}^T\vec{(x_2-x_1)}}
\end{align}
\begin{align}
	\myvec{\vec{m_1}^T\vec{m_1} & \vec{m_1}^T\vec{m_2} & \vec{m_1}^T\vec{(x_2-x_1)}\\\vec{m_2}^T\vec{m_1} & \vec{m_2}^T\vec{m_2} & \vec{m_2}^T\vec{(x_2-x_1)}}
\end{align}
\begin{align}
	\intertext{Given}
	\vec{x_1}=\myvec{8\\-19\\10},\vec{x_2}=\myvec{15\\29\\5},\vec{m_1}=\myvec{3\\-16\\7} ,  \vec{m_2}=\myvec{3\\8\\-5}
	\intertext{The augmented matrix is}
	\myvec{314 & -154 & -782\\ -154 & 98 & 430}\\
	\myvec{314 & -154 &-782\\ -154 & 98 & 430}\xleftrightarrow{R_1=\frac{R_1}{2},R_2=\frac{R_2}{2}}\myvec{157 & -77 &-391\\ -77 & 49 & 215}\\
	\myvec{157 & -77 &-391\\ -77 & 49 & 215}\xleftrightarrow{R_1=R_2+\frac{R_1}{0.49}}\myvec{157 & -77 &-391\\ 0 & 11.24 & 23.24}\\
	\myvec{157 & -77 &-391\\ 0 & 11.24 & 23.24}\xleftrightarrow{R_1=\frac{R_1}{6.85}+R_2}\myvec{22.92 & 0 & -33.84\\ 0 & 11.24 & 23.24}\\
	\myvec{22.92 & 0 & -33.84\\ 0 & 11.24 & 23.24}\xleftrightarrow{R_1=\frac{R_1}{22.92},R_2 =\frac{R_1}{11.24}}\myvec{1 & 0 & -1.48\\ 0 & 1 & 2.07}
\end{align}
\begin{align}
	\lambda_1=-1.48,\lambda_2=2.07
\end{align}
Using the equation $(2.0.2)$ and $(2.0.3)$, we get the points as $A=\myvec{3.56\\4.68\\-0.36}$ and $B=\myvec{8.79\\12.44\\15.35}$ on the line $l_1$,$l_2$ respectively.\par

Direction of line AB $=\myvec{5.22449\\7.83673\\15.6735}$\par
Unit vector along AB
\begin{align}
	\vec{n}=\frac{\vec{A-B}}{\norm{\vec{A-B}}}
\end{align}
Unit vector of AB = $ \myvec{0.285714\\0.428571\\0.857143}$\par
\par

The equation of required line from equation $(2.0.1)$ \par
\begin{align}
	\vec{l_1} =\myvec{1\\2\\-4} + \lambda\myvec{0.285714\\0.428571\\0.857143} 
\end{align}


\begin{figure}[t]
    \centering
    \includegraphics[width =8cm]{Assignment.png}
    \caption{Perpendicular Line }
    \label{fig:1}
\end{figure}
\end{document}